\chapter{Introduction}
\label{chp:introduction}
\epigraph{I \textit{propose} to consider the question, ``Can machines think?''}{Alan Turing - 1950}

Like Turing in the quote cited above, I too propose to consider the question, ``Can machines think?'' Instead of approaching this through a thought experiment as Turing did, nowadays, one can approach this question by asking it to a search engine. When issuing this query to popular web search systems, we get varying results: the first result on Google is a passage generated from the article written by Turing, while the first result on Bing is a passage from a website that concludes machines cannot think\footnote{However, if a machine cannot think, can we trust the result presented by this algorithm?}\footnote{These results were retrieved in October of 2022}.

When looking for \textit{information}, we use systems that process queries every single day. While Google and Bing are all-purpose web engines that mainly focus on finding and retrieving information from the internet, people also use specialized search systems in their day-to-day lives: Amazon and eBay when we are looking for a product to buy, Scholar and ResearchGate for scientific resources, YouTube and TikTok for videos, or Facebook and LinkedIn when we are searching for people. It might even be possible that you are reading this text after you found this document through search. 

When searching for the query, ``Can machines think?'', searching through text documents might be sufficient for the person who searches. However, more than only considering text is needed when someone's information need is more complicated. For example, when one wants to buy a product on Amazon, aspects other than text also need to be considered. Maybe you want to buy an iPhone; information on the price, which edition is the most recent, or which color it has are all essential to determine which one you want. You may also want to consider the rating provided by people who previously bought an iPhone.

If someone searches for people on LinkedIn, they might be more interested in persons they are connected to than strangers. If you are looking for someone to do a job, it is ideal that a shared connection can vouch for them. In this case, how people relate to each other in their network might indicate \textit{relevance}. Not only the structure of how people relate to each other determines relevance; their experience, where they work, or reviews of their previous work will also matter.

Although it might be possible to encode all this information as written text, often, it is more convenient to save this information in a more structured approach. Where information retrieval researchers research the retrieval of ``information'' through text data, data management researchers research the retrieval of structured data~\citep{rijsbergen79information}. This thesis considers both methods simultaneously: systems that can work with structured and unstructured information are investigated.

\section{Problem Description and Research Questions}
Although information retrieval and data retrieval are research fields investigated by different disciplines, they are closely related, and systems that use both have been researched and developed in the past. Techniques developed in one community might help the other, as things like storing and quickly retrieving data are essential for both information and data retrieval. 

The database community has shown an increased interest in graph databases in recent years. As these databases are becoming more popular for data retrieval tasks where the data is highly interconnected, they might also benefit similar tasks in the information retrieval field where data (especially relevant results) is often highly interconnected. This thesis will investigate how these databases, with dedicated graph query languages, can be used for information retrieval tasks. This leads us to this thesis's main research question: \textbf{Research Question: How can information retrieval benefit from graph databases and graph query languages?}

Three sub-research questions are defined to guide us in answering the main research question:

\begin{itemize}
	\item \emph{Research Question 1: What are the benefits of using relational databases for information retrieval?} 
	\item \emph{Research Question 2: Can we extend the benefits from using relational databases for information retrieval to using graph databases while being able to express graph-related problems easier?}  
	\item \emph{Research Question 3: When does information retrieval research benefit from graph data?} 
\end{itemize}

\cref{ir-using-relational-databases,from-tables-to-graphs,ch:mmead} will take these research questions and try to answer them respectively. \cref{a-graph-of-entities} will present work used in \cref{ch:mmead}. Then in~\cref{conclusion}, we will take the answers to these research questions and use them to answer the main research question. 

\section{Thesis Contributions and Structure}

\begin{itemize}
	\item Chapter~\ref{related-work} describes the necessary background information to provide context to the other chapters. The background that is described in this chapter concerns the ``general'' background knowledge for this thesis. Individual chapters also have related work sections that concern the background knowledge in those chapters. The information described in those related work sections contains overlapping information, i.e., the knowledge needed to understand the context of those chapters. This is done to make it possible to understand a chapter without first reading the chapters that come before it. 
	
	\item Chapter~\ref{ir-using-relational-databases} concerns information retrieval research using relational databases. 
	Although, traditionally, inverted indexes are used for information retrieval, we show that by employing relational databases, some advantages are gained. 
	First, attempts to use relational databases for information retrieval through the years will be described. One of the latter attempts used a column-oriented relational database system for information retrieval, achieving competitive efficiency compared to a traditional system built using inverted indexes. This approach is re-implemented as a prototype system, which is then used for a reproduction study. This chapter establishes the usefulness of relational databases for information retrieval by demonstrating its benefits. The content in this chapter is based on the following published works: 
	
	{
		\scriptsize
		\begin{itemize}
			\item \bibentry{olddog-docker}
			\item \bibentry{Kamphuis2020BM25}
		\end{itemize}
	}
	
	\item Chapter~\ref{from-tables-to-graphs} takes the concept of employing relational databases for information retrieval and extends the relational model to the graph model. GeeseDB is introduced, a graph database prototype system built on top of an embedded column-oriented relational engine. Using the graph model makes it possible to express more complex information retrieval problems than when the relational model is used. These more complex models often use multi-stage retrieval approaches. GeeseDB is built on top of an embedded database system, so moving data between ranking stages can be done efficiently. The content of this chapter has previously been described in the following published works:
	
	{
		\scriptsize
		\begin{itemize}
			\item \bibentry{need-graph-db}
			\item \bibentry{geesedb}
		\end{itemize}
	}
	
	\item Chapter~\ref{a-graph-of-entities} introduces the concept of entity linking. Specifically, the Radboud Entity Linker~\citep{rel} system is discussed. When trying to utilize REL to annotate a large corpus with entity linking, issues were found that prohibited the annotations process to such an extent that it was not possible to do within a reasonable time. In order to be able to annotate the corpus, the REL toolkit internals were upgraded, and a batch extension was developed. Altogether this led to more efficient software such that REL can annotate larger corpora. The content in this chapter has previously been described in the following published work: 
	
	{
		\scriptsize
		\begin{itemize}
			\item \bibentry{rebl}
		\end{itemize}
	}
	
	\item Chapter~\ref{ch:mmead} employs the software of chapter~\ref{a-graph-of-entities} to annotate a large web corpus. We developed a specification for sharing entity link annotations. Following this specification, we made annotations for the MS MARCO~\citep{msmarco} corpora publicly available. Using these annotations, we show that, through query expansion, we can increase recall effectiveness for first-stage rankers on this dataset. Next, a demonstration shows how entity links can also be used for geographical information applications. 
	This content has been described in the following published work:
	
	{
		\scriptsize
		\begin{itemize}
			\item \bibentry{mmead}
		\end{itemize}
	}
	
	\item Chapter~\ref{conclusion} serves as a conclusion and summarizes the content discussed in the thesis. We will reflect on the research questions that motivated this dissertation, and discuss what future research is needed.
\end{itemize}

\section{Other Publications}
During the employment at Radboud University, the following work was also published: 

{\scriptsize
	\begin{itemize}	
		\item \bibentry{trec-2019}
		\item \bibentry{ciff}	
		\item \bibentry{trec-2020}
		\item \bibentry{trec-covid}
		\item \bibentry{graphdb-for-ir}
	\end{itemize}
}

Although this work relates to the main subject of the thesis, it is not used as source material for the work described in this thesis. 
\chapter{Introduction}
\epigraph{I \textit{propose} to consider the question, ``Can machines think?''}{Alan Turing - 1950}

I also propose to consider the question, ``Can machines think?'' Instead of approaching this through a thought experiment as Turing did, nowadays, one can approach this question by asking it to a search engine. When issuing this query to popular web search systems, we get varying results: the first result on Google is a passage generated from the article written by Turing, while the first result on Bing is a passage generated from a website that concludes machines can not think\footnote{However, if a machine can not think, can we trust the result presented by this algorithm?}\footnote{These results were retrieved in October of 2022}.

We use systems that process queries daily when looking for \textit{information}. While Google and Bing are all-purpose web engines that mainly focus on finding and retrieving information from the internet, people also use specialized search systems in their day-to-day lives: Amazon and eBay when we are looking for a product to buy, Scholar and ResearchGate for scientific resources, Youtube and TikTok for Videos, or Facebook and LinkedIn when we are searching for people. It might even be possible that you are reading this text after you found this document through search. 

When searching for the query ``Can machines think?'', searching through text documents only might be sufficient for the person who searches. However, more than considering the text is needed when searching today. For example, when one wants to buy a product on Amazon, aspects other than text also need to be considered. Let's say you want to buy an iPhone; information on the price, which edition is the most recent, or which color it has are all essential to determine which one you want. You may also want to consider the rating provided by people that previously bought an iPhone.

If someone searches for people on LinkedIn, they are generally more interested in persons that have connections in common compared to strangers. If you are looking for someone to do a job, it is ideal that a shared connection can vouch for them. In this case, how people relate to each other in their network might indicate \textit{relevance}. Not only the structure of how people relate to each other determines relevance; other examples are their experience, where they work, or reviews of their previous work might matter.

Although it might be possible to encode all this information as written text, often, it is more convenient to save this information in a more structured approach. Where information retrieval researchers research the retrieval of information through text data, the retrieval of structured data is researched by data management researchers. In this thesis, both methods are considered simultaneously: systems that can work with structured and unstructured information are investigated.

\section{Problem Description and Research Questions}
Although information retrieval and data retrieval are research fields investigated by different disciplines, they are closely related, and systems that use both have been researched and developed in the past (In later parts of this thesis, examples are provided). Also, techniques developed in one community might help the other, as things like storing data and quickly retrieving it are essential for information and data retrieval. 

In recent years there has been much exciting research in the database community studying graph databases. What these databases exactly are will be described in chapter \ref{related-work}. As these databases are becoming more popular for data retrieval tasks where the data is highly interconnected, they might also benefit similar tasks in the information retrieval field where data is often highly interconnected. This thesis will investigate how these databases, with dedicated graph query languages, can be used for information retrieval tasks. Leading us to the main research question of this thesis: \textbf{RQ: How can information retrieval benefit from graph databases and graph query languages?}

Three sub-research questions are defined to guide us in answering the main research question:

\begin{enumerate}
	\item \emph{RQ1: What are the benefits of using relational databases for information retrieval?} 
	\item \emph{RQ2: How can we extend the benefits from using relational databases for information retrieval to using graph databases?} 
	\item \emph{RQ3: Can we show practically an example where information retrieval research benefits from graph data?} 
\end{enumerate}

\section{Thesis Contributions and Structure}

\begin{itemize}
\item Chapter~\ref{related-work} will describe all necessary background information to understand the other chapters. 

\item Chapter~\ref{ir-using-relational-databases} will present ..., we will discuss content based on the following published works: \citep{Kamphuis2020BM25, olddog-docker}

\item Chapter~\ref{from-tables-to-graphs} will present ..., we will discuss content which has been previously described in the following published works: \citep{need-graph-db, geesedb}

\item Chapter~\ref{a-graph-of-entities} will present ..., we will discuss content which has been previously described in the following published work: \citep{rebl}

\item Chapter~\ref{conclusion} will serve as a conclusion and tries to summarize the content discussed in the book. Here we will reflect on the research question, and discuss what future research is needed.
\end{itemize}

\section{Publications}

\begin{itemize}
	\item \cite{olddog-docker}
	\item \cite{need-graph-db}
	\item \cite{trec-2019}
	\item \cite{ciff}	
	\item \cite{Kamphuis2020BM25}
	\item \cite{trec-2020}
	\item \cite{trec-covid}
	\item \cite{graphdb-for-ir}
	\item \cite{geesedb}
	\item \cite{rebl}
\end{itemize}

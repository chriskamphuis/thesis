% Required packages:

\chapter*{SIKS Dissertations}
\addcontentsline{toc}{chapter}{SIKS Dissertations}
\markboth{SIKS Dissertations}{SIKS Dissertations}

\begin{xltabular}{\linewidth}{@{} l @{\hspace{0.5em}} l @{\hspace{1em}} X @{}}

% \toprule
2016
	&	 01	&	 Syed Saiden Abbas (RUN), Recognition of Shapes by Humans and Machines\\
	&	 02	&	 Michiel Christiaan Meulendijk (UU), Optimizing medication reviews through decision support: prescribing a better pill to swallow\\
	&	 03	&	 Maya Sappelli (RUN), Knowledge Work in Context: User Centered Knowledge Worker Support\\
	&	 04	&	 Laurens Rietveld (VUA), Publishing and Consuming Linked Data\\
	&	 05	&	 Evgeny Sherkhonov (UvA), Expanded Acyclic Queries: Containment and an Application in Explaining Missing Answers\\
	&	 06	&	 Michel Wilson (TUD), Robust scheduling in an uncertain environment\\
	&	 07	&	 Jeroen de Man (VUA), Measuring and modeling negative emotions for virtual training\\
	&	 08	&	 Matje van de Camp (TiU), A Link to the Past: Constructing Historical Social Networks from Unstructured Data\\
	&	 09	&	 Archana Nottamkandath (VUA), Trusting Crowdsourced Information on Cultural Artefacts\\
	&	 10	&	 George Karafotias (VUA), Parameter Control for Evolutionary Algorithms\\
	&	 11	&	 Anne Schuth (UvA), Search Engines that Learn from Their Users\\
	&	 12	&	 Max Knobbout (UU), Logics for Modelling and Verifying Normative Multi-Agent Systems\\
	&	 13	&	 Nana Baah Gyan (VUA), The Web, Speech Technologies and Rural Development in West Africa - An ICT4D Approach\\
	&	 14	&	 Ravi Khadka (UU), Revisiting Legacy Software System Modernization\\
	&	 15	&	 Steffen Michels (RUN), Hybrid Probabilistic Logics - Theoretical Aspects, Algorithms and Experiments\\
	&	 16	&	 Guangliang Li (UvA), Socially Intelligent Autonomous Agents that Learn from Human Reward\\
	&	 17	&	 Berend Weel (VUA), Towards Embodied Evolution of Robot Organisms\\
	&	 18	&	 Albert Mero\~{n}o Pe\~{n}uela (VUA), Refining Statistical Data on the Web\\
	&	 19	&	 Julia Efremova (TU/e), Mining Social Structures from Genealogical Data\\
	&	 20	&	 Daan Odijk (UvA), Context \& Semantics in News \& Web Search\\
	&	 21	&	 Alejandro Moreno C\'{e}lleri (UT), From Traditional to Interactive Playspaces: Automatic Analysis of Player Behavior in the Interactive Tag Playground\\
	&	 22	&	 Grace Lewis (VUA), Software Architecture Strategies for Cyber-Foraging Systems\\
	&	 23	&	 Fei Cai (UvA), Query Auto Completion in Information Retrieval\\
	&	 24	&	 Brend Wanders (UT), Repurposing and Probabilistic Integration of Data; An Iterative and data model independent approach\\
	&	 25	&	 Julia Kiseleva (TU/e), Using Contextual Information to Understand Searching and Browsing Behavior\\
	&	 26	&	 Dilhan Thilakarathne (VUA), In or Out of Control: Exploring Computational Models to Study the Role of Human Awareness and Control in Behavioural Choices, with Applications in Aviation and Energy Management Domains\\
	&	 27	&	 Wen Li (TUD), Understanding Geo-spatial Information on Social Media\\
	&	 28	&	 Mingxin Zhang (TUD), Large-scale Agent-based Social Simulation - A study on epidemic prediction and control\\
	&	 29	&	 Nicolas H\"{o}ning (TUD), Peak reduction in decentralised electricity systems - Markets and prices for flexible planning\\
	&	 30	&	 Ruud Mattheij (TiU), The Eyes Have It\\
	&	 31	&	 Mohammad Khelghati (UT), Deep web content monitoring\\
	&	 32	&	 Eelco Vriezekolk (UT), Assessing Telecommunication Service Availability Risks for Crisis Organisations\\
	&	 33	&	 Peter Bloem (UvA), Single Sample Statistics, exercises in learning from just one example\\
	&	 34	&	 Dennis Schunselaar (TU/e), Configurable Process Trees: Elicitation, Analysis, and Enactment\\
	&	 35	&	 Zhaochun Ren (UvA), Monitoring Social Media: Summarization, Classification and Recommendation\\
	&	 36	&	 Daphne Karreman (UT), Beyond R2D2: The design of nonverbal interaction behavior optimized for robot-specific morphologies\\
	&	 37	&	 Giovanni Sileno (UvA), Aligning Law and Action - a conceptual and computational inquiry\\
	&	 38	&	 Andrea Minuto (UT), Materials that Matter - Smart Materials meet Art \& Interaction Design\\
	&	 39	&	 Merijn Bruijnes (UT), Believable Suspect Agents; Response and Interpersonal Style Selection for an Artificial Suspect\\
	&	 40	&	 Christian Detweiler (TUD), Accounting for Values in Design\\
	&	 41	&	 Thomas King (TUD), Governing Governance: A Formal Framework for Analysing Institutional Design and Enactment Governance\\
	&	 42	&	 Spyros Martzoukos (UvA), Combinatorial and Compositional Aspects of Bilingual Aligned Corpora\\
	&	 43	&	 Saskia Koldijk (RUN), Context-Aware Support for Stress Self-Management: From Theory to Practice\\
	&	 44	&	 Thibault Sellam (UvA), Automatic Assistants for Database Exploration\\
	&	 45	&	 Bram van de Laar (UT), Experiencing Brain-Computer Interface Control\\
	&	 46	&	 Jorge Gallego Perez (UT), Robots to Make you Happy\\
	&	 47	&	 Christina Weber (UL), Real-time foresight - Preparedness for dynamic innovation networks\\
	&	 48	&	 Tanja Buttler (TUD), Collecting Lessons Learned\\
	&	 49	&	 Gleb Polevoy (TUD), Participation and Interaction in Projects. A Game-Theoretic Analysis\\
	&	 50	&	 Yan Wang (TiU), The Bridge of Dreams: Towards a Method for Operational Performance Alignment in IT-enabled Service Supply Chains\\
	
\midrule
2017
	&	 01	&	 Jan-Jaap Oerlemans (UL), Investigating Cybercrime\\
	&	 02	&	 Sjoerd Timmer (UU), Designing and Understanding Forensic Bayesian Networks using Argumentation\\
	&	 03	&	 Dani\"{e}l Harold Telgen (UU), Grid Manufacturing; A Cyber-Physical Approach with Autonomous Products and Reconfigurable Manufacturing Machines\\
	&	 04	&	 Mrunal Gawade (CWI), Multi-core Parallelism in a Column-store\\
	&	 05	&	 Mahdieh Shadi (UvA), Collaboration Behavior\\
	&	 06	&	 Damir Vandic (EUR), Intelligent Information Systems for Web Product Search\\
	&	 07	&	 Roel Bertens (UU), Insight in Information: from Abstract to Anomaly\\
	&	 08	& 	 Rob Konijn (VUA), Detecting Interesting Differences:Data Mining in Health Insurance Data using Outlier Detection and Subgroup Discovery\\
	&	 09	&	 Dong Nguyen (UT), Text as Social and Cultural Data: A Computational Perspective on Variation in Text\\
	&	 10	&	 Robby van Delden (UT), (Steering) Interactive Play Behavior\\
	&	 11	&	 Florian Kunneman (RUN), Modelling patterns of time and emotion in Twitter \#anticipointment\\
	&	 12	&	 Sander Leemans (TU/e), Robust Process Mining with Guarantees\\
 	&	 13	& 	 Gijs Huisman (UT), Social Touch Technology - Extending the reach of social touch through haptic technology\\
 	&	 14	&	 Shoshannah Tekofsky (TiU), You Are Who You Play You Are: Modelling Player Traits from Video Game Behavior\\
	&	 15	&	 Peter Berck (RUN),  Memory-Based Text Correction\\
	&	 16	&	 Aleksandr Chuklin (UvA), Understanding and Modeling Users of Modern Search Engines\\
	&	 17	&	 Daniel Dimov (UL), Crowdsourced Online Dispute Resolution\\
	&	 18	&	 Ridho Reinanda (UvA), Entity Associations for Search\\
	&	 19	& 	 Jeroen Vuurens (UT), Proximity of Terms, Texts and Semantic Vectors in Information Retrieval\\
	&	 20	&	 Mohammadbashir Sedighi (TUD), Fostering Engagement in Knowledge Sharing: The Role of Perceived Benefits, Costs and Visibility\\
	&	 21	&	 Jeroen Linssen (UT), Meta Matters in Interactive Storytelling and Serious Gaming (A Play on Worlds)\\
	&	 22	&	 Sara Magliacane (VUA), Logics for causal inference under uncertainty\\
	&	 23	&	 David Graus (UvA), Entities of Interest --- Discovery in Digital Traces\\
	&	 24	&	 Chang Wang (TUD), Use of Affordances for Efficient Robot Learning\\
	&	 25	&	 Veruska Zamborlini (VUA), Knowledge Representation for Clinical Guidelines, with applications to Multimorbidity Analysis and Literature Search\\
	&	 26	&	 Merel Jung (UT), Socially intelligent robots that understand and respond to human touch\\
	&	 27	&	 Michiel Joosse (UT), Investigating Positioning and Gaze Behaviors of Social Robots: People's Preferences, Perceptions and Behaviors\\
	&	 28	&	 John Klein (VUA), Architecture Practices for Complex Contexts\\
	&	 29	&	 Adel Alhuraibi (TiU), From IT-BusinessStrategic Alignment to Performance: A Moderated Mediation Model of Social Innovation, and Enterprise Governance of    IT"\\
	&	 30	&	 Wilma Latuny (TiU), The Power of Facial Expressions\\
	&	 31	&	 Ben Ruijl (UL), Advances in computational methods for QFT calculations\\
	&	 32	& 	 Thaer Samar (RUN), Access to and Retrievability of Content in Web Archives\\
	&	 33	&	 Brigit van Loggem (OU), Towards a Design Rationale for Software Documentation: A Model of Computer-Mediated Activity\\
	&	 34	&	 Maren Scheffel (OU), The Evaluation Framework for Learning Analytics \\
	&	 35	&	 Martine de Vos (VUA), Interpreting natural science spreadsheets \\
	&	 36	&	 Yuanhao Guo (UL), Shape Analysis for Phenotype Characterisation from High-throughput Imaging \\
	&	 37	&	 Alejandro Montes Garcia (TU/e), WiBAF: A Within Browser Adaptation Framework that Enables Control over Privacy \\
	&	 38	&	 Alex Kayal (TUD), Normative Social Applications \\
	&	 39	&	 Sara Ahmadi (RUN), Exploiting properties of the human auditory system and compressive sensing methods to increase   noise robustness in ASR \\
	&	 40	&	 Altaf Hussain Abro (VUA), Steer your Mind: Computational Exploration of Human Control in Relation to Emotions, Desires and Social Support For applications in human-aware support systems \\
	&	 41	&	 Adnan Manzoor (VUA), Minding a Healthy Lifestyle: An Exploration of Mental Processes and a Smart Environment to Provide Support for a Healthy Lifestyle\\
	&	 42	&	 Elena Sokolova (RUN), Causal discovery from mixed and missing data with applications on ADHD  datasets\\
	&	 43	&	 Maaike de Boer (RUN), Semantic Mapping in Video Retrieval\\
	&	 44	&	 Garm Lucassen (UU), Understanding User Stories - Computational Linguistics in Agile Requirements Engineering\\
	&	 45	&	 Bas Testerink	(UU), Decentralized Runtime Norm Enforcement\\
	&	 46	&	 Jan Schneider	(OU), Sensor-based Learning Support\\
	&	 47	&	 Jie Yang (TUD), Crowd Knowledge Creation Acceleration\\
	&	 48	&	 Angel Suarez (OU), Collaborative inquiry-based learning\\

\midrule
2018
	&	 01	&	 Han van der Aa (VUA), Comparing and Aligning Process Representations \\
	&	 02	&	 Felix Mannhardt (TU/e), Multi-perspective Process Mining \\
	&	 03	&	 Steven Bosems (UT), Causal Models For Well-Being: Knowledge Modeling, Model-Driven Development of Context-Aware Applications, and Behavior Prediction\\
	&	 04	&	 Jordan Janeiro (TUD), Flexible Coordination Support for Diagnosis Teams in Data-Centric Engineering Tasks \\
	&	 05	&	 Hugo Huurdeman (UvA), Supporting the Complex Dynamics of the Information Seeking Process \\
	&	 06	&	 Dan Ionita (UT), Model-Driven Information Security Risk Assessment of Socio-Technical Systems \\
	&	 07	&	 Jieting Luo (UU), A formal account of opportunism in multi-agent systems \\
	&	 08	&	 Rick Smetsers (RUN), Advances in Model Learning for Software Systems \\
	&	 09	&	 Xu Xie	(TUD), Data Assimilation in Discrete Event Simulations \\
	&	 10	&	 Julienka Mollee (VUA), Moving forward: supporting physical activity behavior change through intelligent technology \\
	&	 11	&	 Mahdi Sargolzaei (UvA), Enabling Framework for Service-oriented Collaborative Networks \\
	&	 12	&	 Xixi Lu (TU/e), Using behavioral context in process mining \\
	&	 13	&	 Seyed Amin Tabatabaei (VUA), Computing a Sustainable Future \\
	&	 14	&	 Bart Joosten (TiU), Detecting Social Signals with Spatiotemporal Gabor Filters \\
	&	 15	&	 Naser Davarzani (UM), Biomarker discovery in heart failure \\
	&	 16	&	 Jaebok Kim (UT), Automatic recognition of engagement and emotion in a group of children \\
	&	 17	&	 Jianpeng Zhang (TU/e), On Graph Sample Clustering \\
	&	 18	& 	 Henriette Nakad (UL), De Notaris en Private Rechtspraak \\
	&	 19	&	 Minh Duc Pham (VUA), Emergent relational schemas for RDF \\
	&	 20	&	 Manxia Liu (RUN), Time and Bayesian Networks \\
	&	 21	&	 Aad Slootmaker (OU), EMERGO: a generic platform for authoring and playing scenario-based serious games \\
	&	 22	&	 Eric Fernandes de Mello Ara\'{u}jo (VUA), Contagious: Modeling the Spread of Behaviours, Perceptions and Emotions in Social Networks \\
	&	 23	&	 Kim Schouten (EUR), Semantics-driven Aspect-Based Sentiment Analysis \\
	&	 24	&	 Jered Vroon (UT), Responsive Social Positioning Behaviour for Semi-Autonomous Telepresence Robots \\
	&	 25	&	 Riste Gligorov (VUA), Serious Games in Audio-Visual Collections \\
	&	 26	& 	 Roelof Anne Jelle de Vries (UT),Theory-Based and Tailor-Made: Motivational Messages for Behavior Change Technology \\
	&	 27	&	 Maikel Leemans (TU/e), Hierarchical Process Mining for Scalable Software Analysis \\
	&	 28	&	 Christian Willemse (UT), Social Touch Technologies: How they feel and how they make you feel \\
	&	 29	&	 Yu Gu (TiU), Emotion Recognition from Mandarin Speech \\
	&	 30	&	 Wouter Beek (VUA),  The "K" in "semantic web" stands for "knowledge": scaling semantics to the web \\
	
\midrule
2019
	&	 01	&	 Rob van Eijk (UL),Web privacy measurement in real-time bidding systems. A graph-based approach to RTB system classification \\
	&	 02	&	 Emmanuelle Beauxis Aussalet (CWI, UU), Statistics and Visualizations for Assessing Class Size Uncertainty \\
	&	 03	&	 Eduardo Gonzalez Lopez de Murillas (TU/e), Process Mining on Databases: Extracting Event Data from Real Life Data Sources \\
	&	 04	&	 Ridho Rahmadi (RUN), Finding stable causal structures from clinical data \\
	& 	 05	&	 Sebastiaan van Zelst (TU/e), Process Mining with Streaming Data \\
	&	 06	& 	 Chris Dijkshoorn (VUA), Nichesourcing for Improving Access to Linked Cultural Heritage Datasets \\
	&	 07	&	 Soude Fazeli (TUD), Recommender Systems in Social Learning Platforms \\
	& 	 08	&	 Frits de Nijs (TUD), Resource-constrained Multi-agent Markov Decision Processes \\
	&	 09	&	 Fahimeh Alizadeh Moghaddam (UvA), Self-adaptation for energy efficiency in software systems \\
	&	 10	&	 Qing Chuan Ye (EUR), Multi-objective Optimization Methods for Allocation and Prediction \\
	&	 11	&	 Yue Zhao (TUD), Learning Analytics Technology to Understand Learner Behavioral Engagement in MOOCs \\
	&	 12	&	 Jacqueline Heinerman (VUA), Better Together \\
	&	 13	&	 Guanliang Chen (TUD), MOOC Analytics: Learner Modeling and Content Generation \\
	&	 14	&	 Daniel Davis (TUD), Large-Scale Learning Analytics: Modeling Learner Behavior \& Improving Learning Outcomes in Massive Open Online Courses \\
	&	 15	&	 Erwin Walraven (TUD), Planning under Uncertainty in Constrained and Partially Observable Environments \\
	&	 16	&	 Guangming Li (TU/e), Process Mining based on Object-Centric Behavioral Constraint (OCBC) Models \\
	&	 17	&	 Ali Hurriyetoglu (RUN),Extracting actionable information from microtexts \\
	&	 18	&	 Gerard Wagenaar (UU), Artefacts in Agile Team Communication \\
	&	 19	&	 Vincent Koeman (TUD), Tools for Developing Cognitive Agents \\
	&	 20	&	 Chide Groenouwe (UU), Fostering technically augmented human collective intelligence \\
	&	 21	&	 Cong Liu (TU/e), Software Data Analytics: Architectural Model Discovery and Design Pattern Detection \\
	&	 22	&	 Martin van den Berg (VUA),Improving IT Decisions with Enterprise Architecture \\
	&	 23	&	 Qin Liu (TUD), Intelligent Control Systems: Learning, Interpreting, Verification\\
	&	 24	&	 Anca Dumitrache (VUA),  Truth in Disagreement - Crowdsourcing Labeled Data for Natural Language Processing\\
	&	 25	&	 Emiel van Miltenburg (VUA), Pragmatic factors in (automatic) image description \\
	&	 26	&	 Prince Singh (UT), An Integration Platform for Synchromodal Transport \\
	&	 27	&	 Alessandra Antonaci (OU), The Gamification Design Process applied to (Massive) Open Online Courses\\
	&	 28	&	 Esther Kuindersma (UL), Cleared for take-off: Game-based learning to prepare airline pilots for critical situations \\
	&	 29	&	 Daniel Formolo (VUA), Using virtual agents for simulation and training of social skills in safety-critical circumstances \\
	&	 30	&	 Vahid Yazdanpanah (UT), Multiagent Industrial Symbiosis Systems \\
	&	 31	&	 Milan Jelisavcic (VUA), Alive and Kicking: Baby Steps in Robotics \\
	&	 32	&	 Chiara Sironi (UM), Monte-Carlo Tree Search for Artificial General Intelligence in Games \\
	&	 33	&	 Anil Yaman (TU/e), Evolution of Biologically Inspired Learning in Artificial Neural Networks \\
	&	 34	&	 Negar Ahmadi (TU/e), EEG Microstate and Functional Brain Network Features for Classification of Epilepsy and PNES \\
	&	 35	&	 Lisa Facey-Shaw (OU), Gamification with digital badges in learning programming \\
	&	 36	&	 Kevin Ackermans (OU), Designing Video-Enhanced Rubrics to Master Complex Skills \\
	&	 37	&	 Jian Fang (TUD), Database Acceleration on FPGAs \\
	&	 38	&	 Akos Kadar (OU), Learning visually grounded and multilingual representations \\

\midrule
2020
	&	 01	&	 Armon Toubman (UL), Calculated Moves: Generating Air Combat Behaviour \\
	&	 02	&	 Marcos de Paula Bueno (UL), Unraveling Temporal Processes using Probabilistic Graphical Models \\
	&	 03	&	 Mostafa Deghani (UvA), Learning with Imperfect Supervision for Language Understanding \\
	&	 04	&	 Maarten van Gompel (RUN), Context as Linguistic Bridges \\
	&	 05	&	 Yulong Pei (TU/e), On local and global structure mining \\
	&	 06	&	 Preethu Rose Anish (UT), Stimulation Architectural Thinking during Requirements Elicitation - An Approach and Tool Support \\
	&	 07	&	 Wim van der Vegt (OU), Towards a software architecture for reusable game components \\
	&	 08	&	 Ali Mirsoleimani (UL),Structured Parallel Programming for Monte Carlo Tree Search \\
	&	 09	&	 Myriam Traub (UU), Measuring Tool Bias and Improving Data Quality for Digital Humanities Research \\
	&	 10	&	 Alifah Syamsiyah (TU/e), In-database Preprocessing for Process Mining \\
	&	 11	&	 Sepideh Mesbah (TUD), Semantic-Enhanced Training Data AugmentationMethods for Long-Tail Entity Recognition Models \\
	&	 12	&	 Ward van Breda (VUA), Predictive Modeling in E-Mental Health: Exploring Applicability in Personalised Depression Treatment \\
	&	 13	&	 Marco Virgolin (CWI), Design and Application of Gene-pool Optimal Mixing Evolutionary Algorithms for Genetic Programming \\
	&	 14	&	 Mark Raasveldt (CWI/UL), Integrating Analytics with Relational Databases \\
	&	 15	&	 Konstantinos Georgiadis (OU),  Smart CAT: Machine Learning for Configurable Assessments in Serious Games \\
	&	 16	&	 Ilona Wilmont (RUN), Cognitive Aspects of Conceptual Modelling \\
	&	 17	&	 Daniele Di Mitri (OU), The Multimodal Tutor: Adaptive Feedback from Multimodal Experiences \\
  	&	 18	&	 Georgios Methenitis (TUD), Agent Interactions \& Mechanisms in Markets with Uncertainties: Electricity Markets in Renewable Energy Systems \\
	&	 19	&	 Guido van Capelleveen (UT), Industrial Symbiosis Recommender Systems \\
	&	 20	&	 Albert Hankel (VUA), Embedding Green ICT Maturity in Organisations \\
	&	 21	&	 Karine da Silva Miras de Araujo (VUA), Where is the robot?: Life as it could be \\
	&	 22	&	 Maryam Masoud Khamis (RUN), Understanding complex systems implementation through a modeling approach: the case of e-government in Zanzibar \\
	&	 23	&	 Rianne Conijn (UT), The Keys to Writing: A writing analytics approach to studying writing processes using keystroke logging \\
	&	 24	&	 Lenin da N\'{o}brega Medeiros (VUA/RUN), How are you feeling, human? Towards emotionally supportive chatbots \\
	&	 25	&	 Xin Du (TU/e), The Uncertainty in Exceptional Model Mining \\
	&	 26	&	 Krzysztof Leszek Sadowski (UU), GAMBIT: Genetic Algorithm for Model-Based mixed-Integer opTimization \\
	&	 27	&	 Ekaterina Muravyeva (TUD), Personal data and informed consent in an educational context \\
	&	 28	&	 Bibeg Limbu (TUD), Multimodal interaction for deliberate practice: Training complex skills with augmented reality \\
	&	 29	&	 Ioan Gabriel Bucur (RUN), Being Bayesian about Causal Inference \\
	&	 30	&	 Bob Zadok Blok (UL), Creatief, Creatiever, Creatiefst \\
	&	 31	&	 Gongjin Lan (VUA), Learning better -- From Baby to Better \\
	&	 32	& 	 Jason Rhuggenaath (TU/e), Revenue management in online markets: pricing and online advertising \\
	&	 33	& 	 Rick Gilsing (TU/e), Supporting service-dominant business model evaluation in the context of business model innovation \\
	&	 34	&	 Anna Bon (UM), Intervention or Collaboration? Redesigning Information and Communication Technologies for Development \\
	&	 35	&	 Siamak Farshidi (UU), Multi-Criteria Decision-Making in Software Production \\

\midrule
2021
	&	 01	&	 Francisco Xavier Dos Santos Fonseca (TUD),Location-based Games for Social Interaction in Public Space \\
	&	 02	&	 Rijk Mercuur (TUD), Simulating Human Routines: Integrating Social Practice Theory in Agent-Based Models \\
	&	 03	&	 Seyyed Hadi Hashemi (UvA), Modeling Users Interacting with Smart Devices \\
	&	 04	&	 Ioana Jivet (OU), The Dashboard That Loved Me: Designing adaptive learning analytics for self-regulated learning \\
	&	 05	&	 Davide Dell'Anna (UU), Data-Driven Supervision of Autonomous Systems \\
	&	 06	&	 Daniel Davison (UT), "Hey robot, what do you think?" How children learn with a social robot \\
	&	 07	&	 Armel Lefebvre (UU), Research data management for open science \\
	&	 08	&	 Nardie Fanchamps (OU), The Influence of Sense-Reason-Act Programming on Computational Thinking \\
	&	 09	&	 Cristina Zaga (UT), The Design of Robothings. Non-Anthropomorphic and Non-Verbal Robots to Promote Children's Collaboration Through Play \\
	&	 10	&	 Quinten Meertens (UvA), Misclassification Bias in Statistical Learning \\
	&	 11	&	 Anne van Rossum (UL), Nonparametric Bayesian Methods in Robotic Vision \\
	&	 12	&	 Lei Pi (UL), External Knowledge Absorption in Chinese SMEs \\
	&	 13	&	 Bob R. Schadenberg (UT), Robots for Autistic Children: Understanding and Facilitating Predictability for Engagement in Learning \\
	&	 14	&	 Negin Samaeemofrad (UL), Business Incubators: The Impact of Their Support \\
	&	 15	& 	 Onat Ege Adali (TU/e), Transformation of Value Propositions into Resource Re-Configurations through the Business Services Paradigm  \\
	&	 16	&	 Esam A. H. Ghaleb (UM), Bimodal emotion recognition from audio-visual cues \\
	&	 17	&	 Dario Dotti (UM), Human Behavior Understanding  from motion and bodily cues using deep neural networks \\
	&	 18	&	 Remi Wieten (UU), Bridging the Gap Between Informal Sense-Making Tools and Formal Systems - Facilitating the Construction of Bayesian Networks and Argumentation Frameworks \\
	&	 19	&	 Roberto Verdecchia (VUA), Architectural Technical Debt: Identification and Management \\
	&	 20	&	 Masoud Mansoury (TU/e), Understanding and Mitigating Multi-Sided Exposure Bias in Recommender Systems \\
	&	 21	&	 Pedro Thiago Timb\'{o} Holanda (CWI), Progressive Indexes \\
	&	 22	&	 Sihang Qiu (TUD), Conversational Crowdsourcing \\
	&	 23	&	 Hugo Manuel Proen\c{c}a (UL), Robust rules for prediction and description \\
	&	 24	&	 Kaijie Zhu (TU/e), On Efficient Temporal Subgraph Query Processing \\
	&	 25	&	 Eoin Martino Grua (VUA), The Future of E-Health is Mobile: Combining AI and Self-Adaptation to Create Adaptive E-Health Mobile Applications \\
	&	 26	& 	 Benno Kruit (CWI/VUA), Reading the Grid: Extending Knowledge Bases from Human-readable Tables \\
	&	 27	&	 Jelte van Waterschoot (UT), Personalized and Personal Conversations: Designing Agents Who Want to Connect With You \\
	&	 28	&	 Christoph Selig (UL), Understanding the Heterogeneity of Corporate Entrepreneurship Programs \\

\midrule
2022
	&	 01	&	Judith van Stegeren (UT), Flavor text generation for role-playing video games \\
	&	 02	&	Paulo da Costa (TU/e), Data-driven Prognostics and Logistics Optimisation: A Deep Learning Journey \\
	&	 03	&	Ali el Hassouni (VUA), A Model A Day Keeps The Doctor Away: Reinforcement Learning For Personalized Healthcare \\
	&	 04	&	\"{U}nal Aksu (UU), A Cross-Organizational Process Mining Framework \\
	&	 05	&	Shiwei Liu (TU/e), Sparse Neural Network Training with In-Time Over-Parameterization \\
	&	 06	& 	Reza Refaei Afshar (TU/e), Machine Learning for Ad Publishers in Real Time Bidding \\
	&	 07	&	Sambit Praharaj (OU), Measuring the Unmeasurable? Towards Automatic Co-located Collaboration Analytics \\
	&	 08	&	Maikel L. van Eck (TU/e), Process Mining for Smart Product Design \\
	&	 09	&	Oana Andreea Inel (VUA), Understanding Events: A Diversity-driven Human-Machine Approach \\
	&	 10	&	Felipe Moraes Gomes (TUD), Examining the Effectiveness of Collaborative Search Engines \\
	&	 11	&	Mirjam de Haas (UT), Staying engaged in child-robot interaction, a quantitative approach to studying preschoolers' engagement with robots and tasks during second-language tutoring \\
	&	 12	&	Guanyi Chen (UU),  Computational Generation of Chinese Noun Phrases \\
	&	 13	&	Xander Wilcke (VUA), Machine Learning on Multimodal Knowledge Graphs: Opportunities, Challenges, and Methods for Learning on Real-World Heterogeneous and Spatially-Oriented Knowledge \\
	&	 14	&	Michiel Overeem (UU), Evolution of Low-Code Platforms \\
	&	 15	&	Jelmer Jan Koorn (UU), Work in Process: Unearthing Meaning using Process Mining \\
	&	 16	&	Pieter Gijsbers (TU/e), Systems for AutoML Research \\
	&	 17	&	Laura van der Lubbe (VUA), Empowering vulnerable people with serious games and gamification \\
	&	 18	&	Paris Mavromoustakos Blom (TiU), Player Affect Modelling and Video Game Personalisation \\
	&	 19	&	Bilge Yigit Ozkan (UU), Cybersecurity Maturity Assessment and Standardisation \\
	&	 20	&	Fakhra Jabeen (VUA), Dark Side of the Digital Media - Computational Analysis of Negative Human Behaviors on Social Media \\
	&	 21	&	Seethu Mariyam Christopher (UM), Intelligent Toys for Physical and Cognitive Assessments \\
	&	 22	&	Alexandra Sierra Rativa (TiU), Virtual Character Design and its potential to foster Empathy, Immersion, and Collaboration Skills in Video Games and Virtual Reality Simulations \\
	&	 23	&	Ilir Kola (TUD), Enabling Social Situation Awareness in Support Agents \\
	&	 24	&	Samaneh Heidari (UU), Agents with Social Norms and Values - A framework for agent based social simulations with social norms and personal values \\
	&	 25	&	Anna L.D. Latour (UL), Optimal decision-making under constraints and uncertainty \\
	&	 26	&	Anne Dirkson (UL), Knowledge Discovery from Patient Forums: Gaining novel medical insights from patient experiences \\
	&	 27	&	Christos Athanasiadis (UM), Emotion-aware cross-modal domain adaptation in video sequences \\
	&	 28	&	Onuralp Ulusoy (UU), Privacy in Collaborative Systems \\
	&	 29	&	Jan Kolkmeier (UT), From Head Transform to Mind Transplant: Social Interactions in Mixed Reality \\
	&	 30	&	Dean De Leo (CWI), Analysis of Dynamic Graphs on Sparse Arrays \\
	&	 31	&	Konstantinos Traganos (TU/e), Tackling Complexity in Smart Manufacturing with Advanced Manufacturing Process Management \\
	&	 32	&	Cezara Pastrav (UU), Social simulation for socio-ecological systems \\
	&	 33	&	Brinn Hekkelman (CWI/TUD), Fair Mechanisms for Smart Grid Congestion Management \\
	&	 34	&	Nimat Ullah (VUA), Mind Your Behaviour: Computational Modelling of Emotion \& Desire Regulation for Behaviour Change \\
	&	 35	&	Mike E.U. Ligthart (VUA), Shaping the Child-Robot Relationship: Interaction Design Patterns for a Sustainable Interaction \\

\midrule
2023
	&	 01	&	Bojan Simoski (VUA), Untangling the Puzzle of Digital Health Interventions \\
	&	 02	&	Mariana Rachel Dias da Silva (TiU), Grounded or in flight? What our bodies can tell us about the whereabouts of our thoughts \\
	&	 03	&	Shabnam Najafian (TUD), User Modeling for Privacy-preserving Explanations in Group Recommendations \\
	&	 04	&	Gineke Wiggers (UL), The Relevance of Impact: bibliometric-enhanced legal information retrieval \\
	&	 05	&	Anton Bouter (CWI), Optimal Mixing Evolutionary Algorithms for Large-Scale Real-Valued Optimization, Including Real-World Medical Applications \\
	&	 06	&	António Pereira Barata (UL), Reliable and Fair Machine Learning for Risk Assessment \\
	&	 07	&	Tianjin Huang (TU/e), The Roles of Adversarial Examples on Trustworthiness of Deep Learning \\
	&	 08	&	Lu Yin (TU/e), Knowledge Elicitation using Psychometric Learning \\
	&	 09	&	Xu Wang (VUA), Scientific Dataset Recommendation with Semantic Techniques \\
	&	 10	&	Dennis J.N.J. Soemers (UM), Learning State-Action Features for General Game Playing \\
	&	 11	&	Fawad Taj (VUA), Towards Motivating Machines: Computational Modeling of the Mechanism of Actions for Effective Digital Health Behavior Change Applications \\
	&	 12	&	Tessel Bogaard (VUA), Using Metadata to Understand Search Behavior in Digital Libraries \\
	&	 13	&	Injy Sarhan (UU), Open Information Extraction for Knowledge Representation \\
	&	 14	&	Selma Čaušević (TUD), Energy resilience through self-organization \\
	&	 15	&	Alvaro Henrique Chaim Correia (TU/e), Insights on Learning Tractable Probabilistic Graphical Models \\
	&	 16	&	Peter Blomsma (TiU), Building Embodied Conversational Agents: Observations on human nonverbal behaviour as a resource for the development of artificial characters \\
	&	 17	&	Meike Nauta (UT), Explainable AI and Interpretable Computer Vision – From Oversight to Insight \\
	&	 18	&	Gustavo Penha (TUD), Designing and Diagnosing Models for Conversational Search and Recommendation \\
	&	 19	&	George Aalbers (TiU), Digital Traces of the Mind: Using Smartphones to Capture Signals of Well-Being in Individuals \\
	&	 20	&	Arkadiy Dushatskiy (TUD), Expensive Optimization with Model-Based Evolutionary Algorithms applied to Medical Image Segmentation using Deep Learning \\
	&	 21	&	Gerrit Jan de Bruin (UL), Network Analysis Methods for Smart Inspection in the Transport Domain \\
	&	 22	&	Alireza Shojaifar (UU), Volitional Cybersecurity \\
	&	 23	&	Theo Theunissen (UU), Documentation in Continuous Software Development \\
	&	 24	&	Agathe Balayn (TUD), Practices Towards Hazardous Failure Diagnosis in Machine Learning \\
	&	 25	&	Jurian Baas (UU), Entity Resolution on Historical Knowledge Graphs \\
	&	 26	&	Loek Tonnaer (TU/e), Linearly Symmetry-Based Disentangled Representations and their Out-of-Distribution Behaviour \\
	&	 27	&	Ghada Sokar (TU/e), Learning Continually Under Changing Data Distributions \\
	&	 28	&	Floris den Hengst (VUA), Learning to Behave: Reinforcement Learning in Human Contexts \\
	&	 29	&	Tim Draws (TUD), Understanding Viewpoint Biases in Web Search Results \\

\midrule
2024
	&	 01	&	Daphne Miedema (TU/e), On Learning SQL: Disentangling concepts in data systems education \\
	&	 02	&	Emile van Krieken (VUA), Optimisation in Neurosymbolic Learning Systems \\
	&	 03	&	Feri Wijayanto (RUN), Automated Model Selection for Rasch and Mediation Analysis \\
	&	 04	&	Mike Huisman (UL), Understanding Deep Meta-Learning \\
	&	 05	&	Yiyong Gou (UM), Aerial Robotic Operations: Multi-environment Cooperative Inspection \& Construction Crack Autonomous Repair \\
	&	 06	&	Azqa Nadeem (TUD), Understanding Adversary Behavior via XAI: Leveraging Sequence Clustering to Extract Threat Intelligence \\
	&	 07	&	Parisa Shayan (TiU), Modeling User Behavior in Learning Management Systems \\
	&	 08	&	Xin Zhou (UvA), From Empowering to Motivating: Enhancing Policy Enforcement through Process Design and Incentive Implementation \\
	&	 09	&	Giso Dal (UT), Probabilistic Inference Using Partitioned Bayesian Networks \\
	&	 10	&	Cristina-Iulia Bucur (VUA), Linkflows: Towards Genuine Semantic Publishing in Science \\
	&	 11	&	withdrawn \\
	&	 12	&	Peide Zhu (TUD), Towards Robust Automatic Question Generation For Learning \\
	&	 13	&	Enrico Liscio (TUD), Context-Specific Value Inference via Hybrid Intelligence \\
	&	 14	&	Larissa Capobianco Shimomura (TU/e), On Graph Generating Dependencies and their Applications in Data Profiling \\
	&	 15	&	Ting Liu (VUA), A Gut Feeling: Biomedical Knowledge Graphs for Interrelating the Gut Microbiome and Mental Health \\
	&	 16	&	Arthur Barbosa Câmara (TUD), Designing Search-as-Learning Systems \\
	&	 17	&	Razieh Alidoosti (VUA), Ethics-aware Software Architecture Design \\
	&	 18	&	Laurens Stoop (UU), Data Driven Understanding of Energy-Meteorological Variability and its Impact on Energy System Operations \\
	&	 19	&	Azadeh Mozafari Mehr (TU/e), Multi-perspective Conformance Checking: Identifying and Understanding Patterns of Anomalous Behavior\\
	&	 20	&	Ritsart Anne Plantenga (UL), Omgang met Regels \\
	&	 21	&	Federica Vinella (UU), Crowdsourcing User-Centered Teams \\
	&	 22	&	Zeynep Ozturk Yurt (TU/e), Beyond Routine: Extending BPM for Knowledge-Intensive Processes with Controllable Dynamic Contexts \\
	&	 23	&	Jie Luo (VUA), Lamarck’s Revenge: Inheritance of Learned Traits Improves Robot Evolution \\
	&	 24	&	Nirmal Roy (TUD), Exploring the effects of interactive interfaces on user search behaviour \\
	&	 25	&	Alisa Rieger (TUD), Striving for Responsible Opinion Formation in Web Search on Debated Topics \\
	&	 26	&	Tim Gubner (CWI), Adaptively Generating Heterogeneous Execution Strategies using the VOILA Framework \\
	&	 27	&	Lincen Yang (UL), Information-theoretic Partition-based Models for Interpretable Machine Learning \\
	&	 28	&	Leon Helwerda (UL), Grip on Software: Understanding development progress of Scrum sprints and backlogs \\
	&	 29	&	David Wilson Romero Guzman (VUA), The Good, the Efficient and the Inductive Biases: Exploring Efficiency in Deep Learning Through the Use of Inductive Biases \\
	&	 30	&	Vijanti Ramautar (UU), Model-Driven Sustainability Accounting \\
	&	 31	&	Ziyu Li (TUD), On the Utility of Metadata to Optimize Machine Learning Workflows \\
	&	 32	&	Vinicius Stein Dani (UU), The Alpha and Omega of Process Mining \\
	&	 33	&	Siddharth Mehrotra (TUD), Designing for Appropriate Trust in Human-AI interaction \\
	&	 34	&	Robert Deckers (VUA), From Smallest Software Particle to System Specification - MuDForM: Multi-Domain Formalization Method \\
	&	 35	&	Sicui Zhang (TU/e), Methods of Detecting Clinical Deviations with Process Mining: a fuzzy set approach \\
	&	 36	&	Thomas Mulder (TU/e), Optimization of Recursive Queries on Graphs \\
	&	 37	&	James Graham Nevin (UvA), The Ramifications of Data Handling for Computational Models \\
	&	 38	&	Christos Koutras (TUD), Tabular Schema Matching for Modern Settings \\
	&	 39	&	Paola Lara Machado (TU/e), The Nexus between Business Models and Operating Models: From Conceptual Understanding to Actionable Guidance \\
	&	 40	&	Montijn van de Ven (TU/e), Guiding the Definition of Key Performance Indicators for Business Models \\
	&	 41	&	Georgios Siachamis (TUD), Adaptivity for Streaming Dataflow Engines \\
	&	 42	&	Emmeke Veltmeijer (VUA), Small Groups, Big Insights: Understanding the Crowd through Expressive Subgroup Analysis \\
	&	 43	&	Cedric Waterschoot (KNAW Meertens Instituut), The Constructive Conundrum: Computational Approaches to Facilitate Constructive Commenting on Online News Platforms \\
	&	 44	&	Marcel Schmitz (OU), Towards learning analytics-supported learning design \\
	&	 45	&	Sara Salimzadeh (TUD), Living in the Age of AI: Understanding Contextual Factors that Shape Human-AI Decision-Making \\
	&	 46	&	Georgios Stathis (Leiden University), Preventing Disputes: Preventive Logic, Law \& Technology \\
	&	 47	&	Daniel Daza (VUA), Exploiting Subgraphs and Attributes for Representation Learning on Knowledge Graphs \\
	&	 48	&	Ioannis Petros Samiotis (TUD), Crowd-Assisted Annotation of Classical Music Compositions \\

\midrule
2025
	&	 01	&	Max van Haastrecht (UL), Transdisciplinary Perspectives on Validity: Bridging the Gap Between Design and Implementation for Technology-Enhanced Learning Systems \\
	&	 02	&	Jurgen van den Hoogen (JADS), Time Series Analysis Using Convolutional Neural Networks \\
	&	 03	&	Andra-Denis Ionescu (TUD), Feature Discovery for Data-Centric AI \\
	&	 04	&	Rianne Schouten (TU/e), Exceptional Model Mining for Hierarchical Data \\
	&	 05	&	Nele Albers (TUD), Psychology-Informed Reinforcement Learning for Situated Virtual Coaching in Smoking Cessation \\
	&	 06	&	Daniël Vos (TUD), Decision Tree Learning: Algorithms for Robust Prediction and Policy Optimization \\
	&	 07	&	Ricky Maulana Fajri (TU/e), Towards Safer Active Learning: Dealing with Unwanted Biases, Graph-Structured Data, Adversary, and Data Imbalance \\
	&	 08	&	Stefan Bloemheuvel (TiU), Spatio-Temporal Analysis Through Graphs: Predictive Modeling and Graph Construction \\
	&	 09	&	Fadime Kaya (VUA), Decentralized Governance Design - A Model-Based Approach \\
	&	 10	&	Zhao Yang (UL), Enhancing Autonomy and Efficiency in Goal-Conditioned Reinforcement Learning \\
	&	 11	&	Shahin Sharifi Noorian (TUD), From Recognition to Understanding: Enriching Visual Models Through Multi-Modal Semantic Integration \\
	&	 12	&	Lijun Lyu (TUD), Interpretability in Neural Information Retrieval \\
	&	 13	&	Fuda van Diggelen (VUA), Robots Need Some Education: on the complexity of learning in evolutionary robotics \\
	&	 14	&	Gennaro Gala (TU/e), Probabilistic Generative Modeling with Latent Variable Hierarchies \\
	&	 15	&	Michiel van der Meer (UL), Opinion Diversity through Hybrid Intelligence \\
	&	 16	&	Monika Grewal (TU Delft), Deep Learning for Landmark Detection, Segmentation, and Multi-Objective Deformable Registration in Medical Imaging \\
	&	 17	&	Matteo De Carlo (VUA), Real Robot Reproduction: Towards Evolving Robotic Ecosystems \\
	&	 18	&	Anouk Neerincx (UU), Robots That Care: How Social Robots Can Boost Children's Mental Wellbeing \\
	&	 19	&	Fang Hou (UU), Trust in Software Ecosystems \\
	&	 20	&	Alexander Melchior (UU), Modelling for Policy is More Than Policy Modelling (The Useful Application of Agent-Based Modelling in Complex Policy Processes) \\
	&	 21	&	Mandani Ntekouli (UM), Bridging Individual and Group Perspectives in Psychopathology: Computational Modeling Approaches using Ecological Momentary Assessment Data \\
	&	 22	&	Hilde Weerts (TU/e), Decoding Algorithmic Fairness: Towards Interdisciplinary Understanding of Fairness and Discrimination in Algorithmic Decision-Making \\
	&	 23	&	Roderick van der Weerdt (VUA), IoT Measurement Knowledge Graphs: Constructing, Working and Learning with IoT Measurement Data as a Knowledge Graph \\
	&	 24	&	Zhong Li (UL), Trustworthy Anomaly Detection for Smart Manufacturing \\
	&	 25	&	Kyana van Eijndhoven (TiU), A Breakdown of Breakdowns: Multi-Level Team Coordination Dynamics under Stressful Conditions \\
	&	 26	&	Tom Pepels (UM), Monte-Carlo Tree Search is Work in Progress \\
	&	 27	&	Danil Provodin (JADS, TU/e), Sequential Decision Making Under Complex Feedback \\
	&	 28	&	Jinke He (TU Delft), Exploring Learned Abstract Models for Efficient Planning and Learning \\
	&	 29	&	Erik van Haeringen (VUA), Mixed Feelings: Simulating Emotion Contagion in Groups \\
	&	 30	&	Myrthe Reuver (VUA), A Puzzle of Perspectives: Interdisciplinary Language Technology for Responsible News Recommendation \\
	&	 31	&	Gebrekirstos Gebreselassie Gebremeskel (RUN), Spotlight on Recommender Systems: Contributions to Selected Components in the Recommendation Pipeline \\
	&	 32	&	Ryan Brate (UU), Words Matter: A Computational Toolkit for Charged Terms \\
	&	 33	&	Merle Reimann (VUA), Speaking the Same Language: Spoken Capability Communication in Human-Agent and Human-Robot Interaction \\
	&	 34	&	Eduard C. Groen (UU), Crowd-Based Requirements Engineering \\
	&	 35	&	Urja Khurana (VUA), From Concept To Impact: Toward More Robust Language Model Deployment \\
	&	 36	&	Anna Maria Wegmann (UU), Say the Same but Differently: Computational Approaches to Stylistic Variation and Paraphrasing \\

% \bottomrule
\end{xltabular}

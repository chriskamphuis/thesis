\documentclass{tufte-book} % Use the tufte-book class which in turn uses the tufte-common class

\hypersetup{colorlinks} % Comment this line if you don't wish to have colored links

\usepackage{microtype} % Improves character and word spacing

\usepackage{lipsum} % Inserts dummy text

\usepackage{booktabs} % Better horizontal rules in tables

\usepackage{graphicx} % Needed to insert images into the document
\graphicspath{{graphics/}} % Sets the default location of pictures
\setkeys{Gin}{width=\linewidth,totalheight=\textheight,keepaspectratio} % Improves figure scaling

\usepackage{fancyvrb} % Allows customization of verbatim environments
\fvset{fontsize=\normalsize} % The font size of all verbatim text can be changed here

\newcommand{\hangp}[1]{\makebox[0pt][r]{(}#1\makebox[0pt][l]{)}} % New command to create parentheses around text in tables which take up no horizontal space - this improves column spacing
\newcommand{\hangstar}{\makebox[0pt][l]{*}} % New command to create asterisks in tables which take up no horizontal space - this improves column spacing

\usepackage{xspace} % Used for printing a trailing space better than using a tilde (~) using the \xspace command

\newcommand{\monthyear}{\ifcase\month\or January\or February\or March\or April\or May\or June\or July\or August\or September\or October\or November\or December\fi\space\number\year} % A command to print the current month and year

\newcommand{\openepigraph}[2]{ % This block sets up a command for printing an epigraph with 2 arguments - the quote and the author
\begin{fullwidth}
\sffamily\large
\begin{doublespace}
\noindent\allcaps{#1}\\ % The quote
\noindent\allcaps{#2} % The author
\end{doublespace}
\end{fullwidth}
}

\newcommand{\blankpage}{\newpage\hbox{}\thispagestyle{empty}\newpage} % Command to insert a blank page

\usepackage{units} % Used for printing standard units

\newcommand{\hlred}[1]{\textcolor{Maroon}{#1}} % Print text in maroon
\newcommand{\hangleft}[1]{\makebox[0pt][r]{#1}} % Used for printing commands in the index, moves the slash left so the command name aligns with the rest of the text in the index 
\newcommand{\hairsp}{\hspace{1pt}} % Command to print a very short space
\newcommand{\ie}{\textit{i.\hairsp{}e.}\xspace} % Command to print i.e.
\newcommand{\eg}{\textit{e.\hairsp{}g.}\xspace} % Command to print e.g.
\newcommand{\na}{\quad--} % Used in tables for N/A cells
\newcommand{\measure}[3]{#1/#2$\times$\unit[#3]{pc}} % Typesets the font size, leading, and measure in the form of: 10/12x26 pc.
\newcommand{\tuftebs}{\symbol{'134}} % Command to print a backslash in tt type in OT1/T1

\providecommand{\XeLaTeX}{X\lower.5ex\hbox{\kern-0.15em\reflectbox{E}}\kern-0.1em\LaTeX}
\newcommand{\tXeLaTeX}{\XeLaTeX\index{XeLaTeX@\protect\XeLaTeX}} % Command to print the XeLaTeX logo while simultaneously adding the position to the index

\newcommand{\doccmdnoindex}[2][]{\texttt{\tuftebs#2}} % Command to print a command in texttt with a backslash of tt type without inserting the command into the index

\newcommand{\doccmddef}[2][]{\hlred{\texttt{\tuftebs#2}}\label{cmd:#2}\ifthenelse{\isempty{#1}} % Command to define a command in red and add it to the index
{ % If no package is specified, add the command to the index
\index{#2 command@\protect\hangleft{\texttt{\tuftebs}}\texttt{#2}}% Command name
}
{ % If a package is also specified as a second argument, add the command and package to the index
\index{#2 command@\protect\hangleft{\texttt{\tuftebs}}\texttt{#2} (\texttt{#1} package)}% Command name
\index{#1 package@\texttt{#1} package}\index{packages!#1@\texttt{#1}}% Package name
}}

\newcommand{\doccmd}[2][]{% Command to define a command and add it to the index
\texttt{\tuftebs#2}%
\ifthenelse{\isempty{#1}}% If no package is specified, add the command to the index
{%
\index{#2 command@\protect\hangleft{\texttt{\tuftebs}}\texttt{#2}}% Command name
}
{%
\index{#2 command@\protect\hangleft{\texttt{\tuftebs}}\texttt{#2} (\texttt{#1} package)}% Command name
\index{#1 package@\texttt{#1} package}\index{packages!#1@\texttt{#1}}% Package name
}}

% A bunch of new commands to print commands, arguments, environments, classes, etc within the text using the correct formatting
\newcommand{\docopt}[1]{\ensuremath{\langle}\textrm{\textit{#1}}\ensuremath{\rangle}}
\newcommand{\docarg}[1]{\textrm{\textit{#1}}}
\newenvironment{docspec}{\begin{quotation}\ttfamily\parskip0pt\parindent0pt\ignorespaces}{\end{quotation}}
\newcommand{\docenv}[1]{\texttt{#1}\index{#1 environment@\texttt{#1} environment}\index{environments!#1@\texttt{#1}}}
\newcommand{\docenvdef}[1]{\hlred{\texttt{#1}}\label{env:#1}\index{#1 environment@\texttt{#1} environment}\index{environments!#1@\texttt{#1}}}
\newcommand{\docpkg}[1]{\texttt{#1}\index{#1 package@\texttt{#1} package}\index{packages!#1@\texttt{#1}}}
\newcommand{\doccls}[1]{\texttt{#1}}
\newcommand{\docclsopt}[1]{\texttt{#1}\index{#1 class option@\texttt{#1} class option}\index{class options!#1@\texttt{#1}}}
\newcommand{\docclsoptdef}[1]{\hlred{\texttt{#1}}\label{clsopt:#1}\index{#1 class option@\texttt{#1} class option}\index{class options!#1@\texttt{#1}}}
\newcommand{\docmsg}[2]{\bigskip\begin{fullwidth}\noindent\ttfamily#1\end{fullwidth}\medskip\par\noindent#2}
\newcommand{\docfilehook}[2]{\texttt{#1}\index{file hooks!#2}\index{#1@\texttt{#1}}}
\newcommand{\doccounter}[1]{\texttt{#1}\index{#1 counter@\texttt{#1} counter}}

\usepackage{makeidx} % Used to generate the index
\makeindex % Generate the index which is printed at the end of the document

% This block contains a number of shortcuts used throughout the book
\newcommand{\vdqi}{\textit{VDQI}\xspace}
\newcommand{\ei}{\textit{EI}\xspace}
\newcommand{\ve}{\textit{VE}\xspace}
\newcommand{\be}{\textit{BE}\xspace}
\newcommand{\VDQI}{\textit{The Visual Display of Quantitative Information}\xspace}
\newcommand{\EI}{\textit{Envisioning Information}\xspace}
\newcommand{\VE}{\textit{Visual Explanations}\xspace}
\newcommand{\BE}{\textit{Beautiful Evidence}\xspace}
\newcommand{\TL}{Tufte-\LaTeX\xspace}

%----------------------------------------------------------------------------------------
%	BOOK META-INFORMATION
%----------------------------------------------------------------------------------------

\title{Graphs and Information Retrieval\thanks{Thanks to Edward R.~Tufte for his inspiration.}} % Title of the book

\author[Chris Kamphuis]{Chris Kamphuis} % Author

\publisher{Publisher of This Book} % Publisher

%----------------------------------------------------------------------------------------
\def\mytitle{Graphs and Information Retrieval}
\def\myauthor{Chris Frans Henri Kamphuis}
\def\mydate{\formatdate{1}{6}{2023}}

% titlepage

\begin{document}
\begin{titlepage}
	\begin{fullwidth}
	\begin{center}
		\vspace*{3.5cm}
		
		%		\LARGE{\textsc{\bfseries\mytitle}}
		\huge{\bfseries\mytitle}
		
		\vspace*{15pt}
		
		%		\large{\textsc{\mysubtitle}}
		% \Large{\mysubtitle}
		
		\vspace*{5pt}
		
		%	    \large{\textsc{More subtitle}}
		\normalsize
		
		\vspace{2.0cm}
		
		Proefschrift
		
		\vspace{0.5cm}
		
		ter verkrijging van de graad van doctor\\
		aan de Radboud Universiteit Nijmegen\\
		op gezag van de rector magnificus prof.~dr.~J.H.J.M.\ van\ Krieken,\\
		volgens besluit van het college van decanen\\
		in het openbaar te verdedigen
		
		\vspace{0.5cm}
		
		op woensdag 22 maart 2023 \\
		om 12:00 uur precies
		
		\vspace{0.5cm}
		
		door
		
		\vspace{0.5cm}
		
		\myauthor\\
		
		geboren op 22 maart 1993 te Oldenzaal, Nederland
	\end{center}
	\end{fullwidth}
\end{titlepage}

\newpage%

% copyright page

\begin{fullwidth}
	\begin{itemize}[leftmargin=*]
		\item[] Promotor:
		\begin{itemize}
			\item[] prof.\ dr.\ ir.\ A. P.\ (Arjen)\ de Vries
		\end{itemize}
	\end{itemize}
	
	\begin{itemize}[leftmargin=*]
		\item[] Manuscriptcommissie:
		\begin{itemize}
			\item[] \makebox[4.2cm]{Person A\hfill} (Affiliation)
			\item[] \makebox[4.2cm]{Person B\hfill} (Affiliation)
			\item[] \makebox[4.2cm]{Person C\hfill} (Affiliation)
		\end{itemize}
	\end{itemize}
	~\vfill
	\thispagestyle{empty}
	\setlength{\parindent}{0pt}
	\setlength{\parskip}{\baselineskip}
	
	\par This work is part of the research program Commit2Data with project number 628.011.001 (SQIREL-GRAPHS), which is (partly) financed by the Netherlands Organisation for Scientific Research (NWO).

	
	Copyright \copyright\ \the\year\ \thanklessauthor
	
	\par\smallcaps{Published by \thanklesspublisher}
	
	\par Licensed under the Apache License, Version 2.0 (the ``License''); you may not use this file except in compliance with the License. You may obtain a copy of the License at \url{http://www.apache.org/licenses/LICENSE-2.0}. Unless required by applicable law or agreed to in writing, software distributed under the License is distributed on an \smallcaps{``AS IS'' BASIS, WITHOUT WARRANTIES OR CONDITIONS OF ANY KIND}, either express or implied. See the License for the specific language governing permissions and limitations under the License.\index{license}
	
	\par\textit{First printing, \monthyear}
\end{fullwidth}

\frontmatter

%----------------------------------------------------------------------------------------
%	DEDICATION PAGE
%----------------------------------------------------------------------------------------

\cleardoublepage
~\vfill
\begin{doublespace}
	\noindent\fontsize{18}{22}\selectfont\itshape
	\nohyphenation
	Dedicated to ...
\end{doublespace}
\vfill
\vfill


\tableofcontents % Print the table of contents

%----------------------------------------------------------------------------------------
% here or at back or not
\listoffigures % Print a list of figures

%----------------------------------------------------------------------------------------
% here or at back or not
\listoftables % Print a list of tables

%----------------------------------------------------------------------------------------
%	EPIGRAPH
%----------------------------------------------------------------------------------------
\newpage
\thispagestyle{empty}
\hspace{0pt}
\vfill
\begin{center}
	\openepigraph{I PROPOSE to consider the question, ``Can machines think?''}{Alan Turing - 1950}
\end{center}
\vfill
\hspace{0pt}


\mainmatter

%----------------------------------------------------------------------------------------
%	INTRODUCTION
%----------------------------------------------------------------------------------------

\cleardoublepage
\chapter{Introduction}

% In this chapter we first introduce the fields of Information Retrieval and Databases, IR focusses on the data management and processing for search engines, whereas Database Research investigates the processing of general structured data. Information Retrieval researchers often use their own data structures, which are specialized for IR applications. We compare this to what is often done at database research, and show examples where these fields overlap. 

I also propose to consider the question, ``Can machines think?'' Instead of approaching this through a thought experiment like Turing\cite{Turing-Think} did, nowadays one can approach this question by asking it to a search engine. When issuing this query to popular web search engines we get different results; the first result on Google is a passage generated from the article written by Turing, while the first result on Bing is a passage generated from a website that states machines can not think\footnote{However, if a machine can not think, can we trust the result presented by this algorithm?}.
We use these systems that process queries every day in our lives to provide us information. Whereas Google and Bing are all purpose web engines that mainly focus on finding and retrieving web data, people also used specialized search systems in their day-to-day lives, examples are: Amazon / EBay for product search, NS for public transport in the Netherlands, Scholar / Zeta Alpha for scientific resources, Youtube / TikTok for Videos, or Facebook / LinkedIn for people.
When searching for the query ``Can machines think?'', the approach of searching through text document only might be sufficient for the user. However in many cases when searching today, only considering text is not sufficient. When one wants to buy a product on Amazon, aspects other than text also need to be considered. Lets say for example you want to buy an IPhone; What is the price, which edition is the most recent, or which color does it have. When someone searches for people on LinkedIn, they are generally more interested in persons that have connections in common compared to complete strangers. If you are looking for someone to do a job, it is ideal that a shared connection can vouch for them. 

\section{Information Retrieval}
Everything that is needed to process a query like, ``Can machines think?'', is subject to research by the field of information retrieval. 

\subsection{Inverted Indexes}

\section{Relational Databases}
Relational databases are usually used to store structure data. 

\section{Graphs}
Instead of using columnar data, it might be more attractive to model your data using graphs. 

%----------------------------------------------------------------------------------------
%	CHAPTER 1
%----------------------------------------------------------------------------------------

\chapter{Information Retrieval using Relational Databases}
\section{Introduction}
Where commonly information retrieval researchers use inverted indexes as data structures, there is also a rich history of researchers using relational databases for representing the data in information retrieval systems. 

% Describe boolean retrieval systems
% Describe other systems by for example Norbert 

In a more recent work M\"{u}hleisen et al.~showed that the common used BM25 ranking function can also be easily expressed using relational tables. Their work specifically focused on the retrieval efficiency of several systems.   
% End with OldDog system by Hannes

\section{Reproducibility}


\chapter{From Tables to Graphs}

% This chapters should describe the GeeseDB paper

\chapter{Data modeling using Graphs}

\chapter{Applications}


\newpage
\thispagestyle{empty}
\hspace{0pt}
\vfill
\begin{center}
	\openepigraph{We can only see a short distance ahead, but we can see plenty there that needs to be done.}{Alan Turing}
\end{center}
\vfill
\hspace{0pt}


\chapter[Conclusion]{Conclusion}
\label{ch:conclusion}

We finalize 
%------------------------------------------------

%----------------------------------------------------------------------------------------

\backmatter

%----------------------------------------------------------------------------------------
%	BIBLIOGRAPHY
%----------------------------------------------------------------------------------------

\bibliography{bibliography} % Use the bibliography.bib file for the bibliography
\bibliographystyle{plainnat} % Use the plainnat style of referencing

%----------------------------------------------------------------------------------------

\chapter*{Summary}
In this chapter

\chapter*{Samenvatting}
In dit hoofdstuk 

\chapter*{Acknowledgements}
I would like to acknowledge ..

\chapter*{Research Data Management}
For RDM

\chapter*{Curriculum Vitæ}
Chris is 




% keep or not
% \printindex % Print the index at the very end of the document

\end{document}
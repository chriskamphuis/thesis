\chapter{Conclusion}
\label{conclusion}
\epigraph{We can only see a short distance ahead, but we can see plenty there that needs to be done.}{Alan Turing - 1950}

 This thesis will be concluded using the research questions as a guide. We will look once more at the work presented in the individual chapters, and tie it together with the research questions. Finally, I want to reflect on the work described in this thesis and present research opportunities that follow this work.

\section{Contributions}
Recall the main research question and its subquestions:
\begin{itemize}
	\item \textbf{Research Question: How can information retrieval benefit from graph databases and graph query languages?}
	\item \emph{Research Question 1: What are the benefits of using relational databases for information retrieval?} 
	\item \emph{Research Question 2: Can we extend the benefits from using relational databases for information retrieval to using graph databases while being able to express graph-related problems easier?} 
	\item \emph{Research Question 3: When does information retrieval research benefit from graph data?} 
\end{itemize}

\cref{ir-using-relational-databases} introduces the history of using relational databases for information retrieval. One of the latter attempts, using columnar databases for retrieval, is highlighted. This approach is re-implemented as a prototype system: ``OldDog''. OldDog is used for a reproduction experiment, where many expressions of BM25 are compared against each other. Because OldDog was built using a relational database, we could express the logical model of the ranking functions separately from their physical models. This way, we could find the differences between models while keeping the data the same. This work demonstrates the usefulness of using relational databases for reproduction experiments in information retrieval. Looking at research question 1: \emph{What are the benefits of using relational databases for information retrieval?}, we have demonstrated that relational databases provide a framework for easily comparing different ranking methods, as shown in the reproduction study. The system is reasonably efficient, making it easy to set up for new experiments. 

\cref{from-tables-to-graphs} takes the data model for information retrieval using relational databases, as presented in~\cref{ir-using-relational-databases}, and extends this model to a graph data model. This model is implemented in GeeseDB, a prototype graph database for information retrieval. GeeseDB is built on top of DuckDB and uses its column-oriented tables and the fact that it can be run in-process. With a robust backend, we can express graph queries for information retrieval and run them on GeeseDB. GeeseDB supports all functionalities that OldDog already supports, plus it makes the expression of more complicated problems through the graph framework easier. Considering research question 2: \emph{Can we extend the benefits from using relational databases for information retrieval to using graph databases, while being able to express graph-related problems easier?}, we find that, with GeeseDB, it is possible to take all the benefits from using relational databases for information retrieval, while making it possible to express more complex problems through a graph query language.

\cref{a-graph-of-entities} concerns improvements on the Radboud Entity Linking (REL) system. Some unforeseen issues were encountered when trying to deploy this system to annotate a large document collection. The time it took to tag the whole collection was much larger than expected, making it unfeasible to annotate the whole corpus in a reasonable time. By improving the efficiency of several components of the software and including a batch extension, it was possible to speed up parts of the software by a factor of ten. After deploying these improvements, it was possible to use the REL software for large corpora. 

\cref{ch:mmead} continues with the system created in \cref{a-graph-of-entities} and deploys it to annotate the large MS MARCO document and passage collections. A specification on how to share these annotations, independent of the linking system, is proposed. Following this specification, we make the annotations created by the REL system publicly available. We also publish data created by the BLINK entity linking system for the MS MARCO v1 passage collection. Using the annotations by REL, we show on the MS MARCO v1 collection that query expansion with entity annotations significantly improves recall for complicated queries. Although this is just a simple experiment, we show that these annotations contain valuable information for finding relevant documents, and they could be beneficial for more sophisticated methods. Also, through a demonstration that combines entity annotations with geographical information, we show that entity annotations benefit interactive search applications—considering research question 3: \emph{When does information retrieval research benefit from graph data?}. We demonstrate that information retrieval can benefit from graph data, such as an entity graph. Employing it for retrieval techniques leads to an increase in retrieval effectiveness. Also, the graph model is ideal for interactive search systems. 

Finally, to address the main research question; \textbf{How can information retrieval benefit from graph databases and graph query languages?}. This thesis demonstrates the usefulness of graph databases for information retrieval by creating a prototype system, GeeseDB, that is directly useful for information retrieval. With GeeseDB, it is possible to set up retrieval experiments using the graph model quickly. It inherits all benefits of using relational databases for information retrieval. We show that we can increase retrieval effectiveness by using a graph of entities. This graph of entities can improve interactive search applications as well.    

\section{Future Work}
Following the work presented in this thesis, interesting research opportunities lie ahead. In particular, I would like to highlight three directions that seem promising to improve retrieval:

Firstly, an obvious follow-up on the work presented in this thesis is a study that combines the work of~\cref{from-tables-to-graphs,ch:mmead}. In~\cref{from-tables-to-graphs}, we introduced GeeseDB, a graph-based system for information retrieval. In~\cref{ch:mmead}, we use graph data produced by the REL system.  It would be interesting to see if we can reproduce the results of~\cref{ch:mmead} using GeeseDB. We used an SPARQL engine for the demonstration in~\cref{ch:mmead}. While convenient for the purpose of this demonstration, it would also be interesting to see how well GeeseDB would hold up in terms of effiency compared to the SPARQL engine. 

The examples shown in~\cref{from-tables-to-graphs} are relatively straightforward. Additional studies should be carried out that consider more complex graph structures. For example, can we employ a graph structure and utilize paths in the graph? These paths could propagate probabilities used to calculate relevancy scores, a strategy that is naturally expressed using graphs.

Lastly, it would be interesting to see if the concepts of graph databases for information retrieval and deep learning for information retrieval can be married together. In the last couple of years, deep learning for information retrieval has become more prominent, especially with the rise of large language models. The representations found by these models can be expressed in graph databases, and both techniques can benefit from each other. Additional studies need to be conducted to determine how this can be done well. 
